\documentclass[conference]{IEEEtran}
\usepackage{cite}
\usepackage{amsmath,amssymb,amsfonts}
\usepackage{algorithmic}
\usepackage{graphicx}
\usepackage{textcomp}
\usepackage{xcolor}
\usepackage{float}
\usepackage{tikz, pgfplots}
\usepackage{circuitikz}
\usepackage[english]{babel}
\usepackage[autostyle, english = american]{csquotes}
\MakeOuterQuote{"}
\def\BibTeX{{\rm B\kern-.05em{\sc i\kern-.025em b}\kern-.08em
    T\kern-.1667em\lower.7ex\hbox{E}\kern-.125emX}}

\pgfplotsset{compat=newest}

\begin{document}

\title{Heartrate Monitor\\

\author{\IEEEauthorblockN{Tevin Hendess}
\IEEEauthorblockA{\textit{Computer Engineering Department} \\
\textit{Rochester Institute of Technology}\\
Rochester, NY USA \\
twh4619@rit.edu}
\and
\IEEEauthorblockN{Adam Schultzer}
\IEEEauthorblockA{\textit{Computer Engineering Department} \\
\textit{Rochester Institute of Technology}\\
Rochester, NY USA \\
ajs1539@rit.edu}
}
}

\maketitle

\begin{abstract}
Knowing the heartrate of a human being is something which is
necessary for many medical applications and is useful for knowing
one's overall health. Using very simple components, a rudimentary
heartrate sensor can be created. The most challenging aspect is ensuring
that the small biological changes are detected and translated into
electrical signals which can be parsed and used for computations. This
requires a number of filters and amplifiers to remove excess noise,
increase the detectability of useful information, and properly delineate
changes in heart beats. Using an OPB745 light sensor in conjunction
with a band-pass filter and three op-amp circuits, a human's heartrate
can be recorded and printed by a computer program.
\end{abstract}

\section{Background}
\subsection{Sensor Functionality}
The method by which a heartbeat used in this case is called PPG
(Photoplethysmogram). This method uses an optical method to detect
changes in blood volume within a patient's tissue. Most often, this is
measured on a finger, for ease of use and application. In this instance,
the OPB745 sensor was used.

\subsection{Sensor Methodology}
The method used for measuring change in blood volume involves shining an
infrared LED into the finger, and measuring the amount of that light
reflected out of the finger. This is represented as the voltage across the
phototransistor within the OPB745.

\section{Design Methodology}
    \subsection{Design Choices}
    \subsection{Relevant Equations}
    \subsection{Schematics}
    \subsection{Transfer Functions}

\section{Results and Analysis}
    \subsection{Simulation Magnitude and Phase Plots}
    \subsection{Oscilloscope and PuTTY Captures}
    \subsection{Comparison and Explanation of Measured vs Simulated}

\section{Conclusion}

\end{document}
